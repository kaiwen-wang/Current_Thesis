\documentclass[12pt, letterpaper, oneside]{book}

% 1. Page Layout
\usepackage[margin=1.5in]{geometry} % Standard 1-inch margins
\usepackage[utf8]{inputenc}

% 2. Packages for your Graphs & Images
\usepackage{graphicx}
\usepackage{pgfplots}
\pgfplotsset{compat=1.18}
\usepackage{tikz}

% 3. Metadata
% \title{Evaluation of LLMs in Generation of Primitive Components for Frontend Design Systems}
\title{The Applications of Website}
\author{Kaiwen Wang}
\date{December 12, 2025}

\begin{document}

% --- FRONT MATTER (Pages are numbered i, ii, iii) ---
\frontmatter

\maketitle


% \chapter{Foreword}

% I have aimed to create something not too wordy.


\tableofcontents

% --- MAIN MATTER (Pages restart at 1, 2, 3) ---
\mainmatter

\chapter{Introduction}

% As humans move up the technology tree, things which were previously novel become repeatable subunits for the next layer. For example, there is little more understanding needed to innovate screws---most shapes exist and engineering projects select from an existing set of them. To not 'reinvent the wheel' is a comparative advantage for any organization.


This thesis is about technology, or more specifically, frontend design systems in Vue with a corollary on LLMs and AI, but we start with an analogy to set the pace.

In our day and age we are presented with myraid of choices; we may see numerous brands of fast foods: McDonalds, Burger King, Wendy's, or one might go to a supermarket and see countless options of foods and brands, or even further one might see numerous styles and brands of clothes but the underlying stitching and fabric material remains the same.

% yet all of this is an illusion.

Because fast food restaurants must work under constraints such as the availability of foodstuffs, of which mass-produced vegetable oils and factory farmed chicken has the cheapest scale, each place is essentially indistinguishable from the other only in that they differ slightly in their appearance and pricing. Restaurants at the low end in effect are a wrapper around food wholesalers such as Costco Business Center.

Processed foods in a supermarket are a wrapper around bleached wheat as per USDA standards and vegetable oils: look at the ingredients list and the all too familiar "Enriched Flour (Wheat Flour, Niacin, Reduced Iron, Vitamin B1 [Thiamin Mononitrate], Vitamin B2 [Riboflavin], Folic Acid)" plus "Soybean Oil" appears. Any sense of difference is an illusion once the wrapper is closely scrutinized.

This tendency of seeming differentiation but implied homogeneity is characteristic of the modern times, and the ability to see past this false differentiation is necessary to not get caught in a vortex of confusion, for there lies infinite combination of items one can find themselves looking at and trapped. Likewise, infinite incremental improvements can always be made as well, but without a core which is

Instead, a sort of piercing ability to get at the core and essence of reality is needed: to hack at the overgrown jungle and see the straight path ahead. A trusty tool to wield in this is writing a chronology of how things were, how they came to be, and the reasons and rationale for where it shall be going. By doing so, holes or gaps can be identified in the current paradigm, giving motive for action.

Many industries follow this pattern: growth and speciation as a frontier opens and consolidation as the frontier closes.





% This thesis aims to be multipart: not just a summary of a summary of things, but an explanation of how things came to be: reasons and rationale for it and where it shall be going. By doing so, we identify holes or gaps in the current paradigm, which leads to the experimental part of this thesis.

% Something that currently frustrates me with current LLM website builders is that they use a default paradigm: build a general sketch of the website using existing design system components, but we have no idea of the internals of the design system, whether it looks good and is appropriate, nor is it so easy to change later. Thus such a builder is unsatisfactory for my craft which aims to develop high quality websites which are understandable---furthermore, the proliferation of numerous design systems of which I am unsure of the internals and usage of.

% So this thesis will dissect these design systems: what are their underlying principles? What makes them up?




\section{The Development of Technologies}

All technologies are implemented by humans, who together through shared mutual understanding, are then able to create themselves or have others implement these mental abstractions. What we see on a computer is no different: the specification bodies are the W3C for CSS and accessibility, WHATWG for HTML, and Ecma International for Javascript. Google, Apple, and Mozilla then respectively implement the Blink, WebKit, and Gecko rendering engine.

Nevertheless, these specifications are rarely enough to satisfy two significant demands of frontend: reactivity and composability. Reactivity is the automatic response in appearance to the change in a variable or internal state while composability means the ability to break a website down into pieces so as to develop parts of the website without interfering in other parts in addition to reusing pieces.

% Add why this is relevant

Currently and seemingly for the near future, frontend web frameworks such as React, Vue, Svelte, Solid, and Angular remain predominant for website development to fulfill these two requirements well. To some extent Web Components and Lit may allow reactivity and composability, but their awareness and usage in the developer community is not as high and the extent of their ecosystems not as developed, so for the sake of brevity we can evaluate whether these options are suitable for development at another time.

At an even higher level include features such as modals, buttons, breadcrumbs, sidebars, tabs, and other components which are commonly tabulated in a 'design system' due to their frequent reuse across web apps. When components have no CSS to them, they are termed 'headless components'.

Styling these days is generally scoped to the component with Tailwind CSS which gives numerous small design tokens such as 'mt-2' as shorthand for "margin-top: 2rem," CSS Modules if one prefers writing native CSS, or it is scoped to the component using the framework. Vue or Svelte's Single-File Components do this. A solution that used to be more common in React but has since lost ground to Tailwind is CSS-in-JS with Emotion or styled-components, as it tended to be somewhat verbose and messy syntax.

Finally, only after considering all these base layers, can a software developer then consider the relation of the frontend to the browser with full-stack frameworks such as Nuxt, Next.js, SvelteKit, client and server-side rendering, its relation to backend servers and databases, and fundamentally the software's relation to the real world and its business uses.

\subsection{Repeatable Subunits}

As humans move up the technology tree, things which were previously novel become repeatable subunits for the next layer. For example, there is little more understanding needed to innovate screws---most shapes exist and engineering projects select from an existing set of them. Electricity and steel become commonplace. To not 'reinvent the wheel' is a comparative advantage for any organization.


We can make the observation, then, that no well-designed website can exist without proper layouts, and that no layouts can be good without proper components, of which no component would be good without base-level features such as properly designed buttons.


If one were to design a web app or mobile app today, they would not reinvent the browser or the phone, and likely not frontend frameworks either. But upon trying to build a web app, one would either try to pick a design system haphazardly or invent their own.


Building iOS and MacOS apps can often be easier than building web apps or React Native as SwiftUI already creates reasonably good looking and reusable components. Among current design system components for the web, none seem to stand out as a clear leader or duopoly in the way that Chromium or React/Vue is. It may be that design systems simply do not have a high complexity ceiling as the total number of commits graphed in comparison to other major parts of the stack are. However, creating apps for Apple are often slower because of the need to build them, whereas hot reloading is a feature of React Native and web frontends.







% The environment in which design systems find themselves is crowded and one purpose of this thesis will be to find order in them.


\subsection{AI does not create repeatable subunits}

My purpose in explaining these levels is to show the iterative and accumulative processes of technology in contrast to the 'Wunderwaffe' or 'magical weapon/cure-all/fix-all' theory of AI.

Based on the Github commits of projects in this stack, it is unlikely that current AI models will achieve the capability to build an entirely new browser, frontend web framework, or one-shot a website from nothing much less build entire cities.


Many AI website builders such as Lovable, v0, or Bolt are often used to create entire websites from a prompt, where further changes are made by inspecting the app visually and requesting further changes through text. These in my experience create unmaintainable websites despite initially satisfactory appearing interfaces for a few reasons: (1) appearing overly generic, (2) the amount of code generated exceeds the cognitive ability of the person who generated it to understand it, maintain it, and extend it (a) in both its usage and (b) the origin and nature of the dependencies used.


The reason that appearing overly generic is a problem is because dimensionality reduction destroys important yet pertinent information. It is easier to distinguish quality between three different software products on their website than three different Github READMEs, just as it is easier to distinguish people based on their overall mannerisms and tendencies than a text biography. Therefore, generating websites which are not distinguished qualitatively presents a negative signaling indicator to users of that product, where the benefit of the trend inversely scales with the amount of people who are able to do it.

The cognitive downside of AI website builders mainly presents itself in structure and components.
Here is a colloquial dialogue of why AI in its current usage fails to meet quality frontend needs: "ok, you try to build something with AI, but it's crappy. So then you try to use a design library you don't really understand but hope to understand. That's still not much better, and you're lost in the components. You then try to one-shot all your own components for a design system, but that's inconsistent. So this thesis standardizes the different components, they are written out, tell people what is necessary for each of them, and see how good LLMs are at making them."

We assume the browser---Chronium, frontend web frameworks---React, Vue, etc. as given, and focus on understanding and solidfying the next level up (design systems/components) rather than immediately focusing on the website/end-product/user layer that lies above that.



\subsection{The Nature of AI}

From a distant and teleological future perspective, it does matter greatly whether AI is plant-like or animal-like. Should it be plant-like, humans are able to harvest the bounties of its complexity as one throws potatoes into the ground and gets food for free. Yet if its inherent nature is animal-like with its own internal state and opinion of the world and what it is to be, it would be extreme hubris to believe something stronger than man would ever work for others or be controlled, and any attempt at doing so would likely provoke its wrath. In its current form it seems to be neither: merely a tool that extends the ability of man and it is from this perspective we will analyze how we can build upon the current usage in frontend design and why its application to design systems is novel and beneficial, addressing a gap in current awareness.



\chapter{Design System Clarified}
\section{Vue or React?}

Currently we live in an age where technology is subject to the whims of the geopolitical systems. We must consider, then, that the choices people make are path-dependent and lock them into a certain world or ecosystem.

By world and ecosystem, I mean the patterns of interlocking norms, systems, tools, words and mannerisms unique to an area.

I believe that the maintainers of React possess an ethnic chauvinism included to: believing English as a language that stands alone amongst all, Guillermo Rauch's endorsement of Benjamin Netanyahu, plus the constant acquisitions of other works in this space such as Svelte or Nuxt, bringing them more in line with React conventions such as state or effect, whereas before it allowed one to declare the variable directly.





\section{List of design systems}

There is a list of design systems which are known in Vue: PrimeVue, ElementPlus, Quasar, shadcn-vue, Nuxt UI, Radix Vue, Vuetify, Naive UI, Ant Design Vue, Arco Design, TDesign,


\section{The essence of a design system}

Now comes the hard part of understanding design systems: give a field with many different species, come together to find a taxonomy


\section{Do design systems have subcategories?}

One common practice is giving a bunch of components in an alphabetical list. Another one, which is more common in Chinese Vue systems is categorizing them.

1. Here is a list of attempting to cateogrize component.gallery


2. We know first of all there is the input, state, and output categorization which can be done to humans themselves.


3. Or, here is the attempt ot categorizing component.gallery using the Chinese style


\section{Which design system components are replicated by native browser features?}



\backmatter
% \bibliography{references}

\end{document}